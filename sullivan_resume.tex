% LaTeX file for resume 
% This file uses the resume document class (res.cls)
\documentclass[10pt]{article}
\usepackage[T1]{fontenc}
\usepackage[latin1]{inputenc}
\usepackage{CormorantGaramond}
\usepackage[english]{babel} % English hyphenation, etc.
% Note: may need to includehead in geometry package
\usepackage[top=1.2in, bottom=.8in, left=0.8in, right=0.8in, headheight=44pt, headsep=0pt]{geometry}


%%%%%%%%%%%%%%%%%%%%%%%%%%%%%%%%%%%%%%%%%%%%%%%%%%%%%%%%%%%%%%%%%%%%%%%%%%%%%%
% Personal info
\newcommand{\myemail}{\href{mailto: tasulliv@ucsd.edu}{tasulliv@ucsd.edu}}
\newcommand{\myphone}{(347) 669-2103}

%%%%%%%%%%%%%%%%%%%%%%%%%%%%%%%%%%%%%%%%%%%%%%%%%%%%%%%%%%%%%%%%%%%%%%%%%%%%%%
% Spacing

% helpful indent values
\newcommand{\fixw}{18pt}
\newcommand{\mintextwidth}{\textwidth - \fixw - \fixw}

% Adjust spacing between lines
\usepackage{setspace}
% \setstretch{1.15}

% Change space between paragraphs
\usepackage{parskip}
\setlength{\parskip}{0pt}
% \setlength{\baselineskip}{0pt}

 % remove all indents 
\setlength\parindent{0pt}

% Some standardized vertical spacing
\newcommand{\mysep}{\vspace{4pt}}

%%%%%%%%%%%%%%%%%%%%%%%%%%%%%%%%%%%%%%%%%%%%%%%%%%%%%%%%%%%%%%%%%%%%%%%%%%%%%%
% Tables

% allows you to set tabular width equal to text width
\usepackage{tabularx}
% to use multicolumn
\usepackage{booktabs}

% remove space before and after table columns
% Don't need to write \begin{tabularx}{\textwidth}{@{}Xr@{}}
% Can just write \begin{tabularx}{\textwidth}{Xr}
\setlength{\tabcolsep}{0pt} 


% Create a left aligned column .8 of textwidth 
\newcolumntype{L}{>{\raggedright\arraybackslash}p{.8\textwidth}}
% Create a right aligned column .2 of textwidth
\newcolumntype{R}{>{\raggedleft\arraybackslash}p{.2\textwidth}}


% Create a multicolumn that is text width minus a tab
% Calculate length
\usepackage{calc}
\newlength{\twtab}
\setlength{\twtab}{\textwidth-\fixw} % find textwidth minus tab 
% With a \vspace
% \newcommand{\tab}[1]{\noalign{\vspace{3pt}} \multicolumn{2}{@{\hspace{\fixw}}p{\twtab}}{{#1}}}
% Without a vspace
\newcommand{\tab}[1]{\multicolumn{2}{@{\hspace{\fixw}}p{\twtab}}{{#1}}}
% Without indent:
% \newcommand{\tab}[1]{\noalign{\vspace{3pt}} \multicolumn{2}{@{}p{\textwidth}}{{#1}}}


%%%%%%%%%%%%%%%%%%%%%%%%%%%%%%%%%%%%%%%%%%%%%%%%%%%%%%%%%%%%%%%%%%%%%%%%%%%%%%


%%%%%%%%%%%%%%%%%%%%%%%%%%%%%%%%%%%%%%%%%%%%%%%%%%%%%%%%%%%%%%%%%%%%%%%%%%%%%%
% Colors
\usepackage[dvipsnames]{xcolor}
\definecolor{MainBlue}{RGB}{0, 106, 150} % UCSD's main blue
% set up hyperlink's
\usepackage[plainpages=false,pdfpagelabels]{hyperref}
\hypersetup{
  colorlinks   = true, %Colours links instead of ugly boxes
  urlcolor     = MainBlue, %Colour for external hyperlinks
  linkcolor    = MainBlue, %Colour of internal links
  citecolor   = red %Colour of citations
}
%%%%%%%%%%%%%%%%%%%%%%%%%%%%%%%%%%%%%%%%%%%%%%%%%%%%%%%%%%%%%%%%%%%%%%%%%%%%%%

% Create bulleited list of fixed width
\usepackage{enumitem}
\newlist{blist}{itemize}{2}
\setlist[blist,1]{label=\textendash,left=0pt .. \fixw}
\setlist[blist,2]{label=\textendash,left=0pt .. \fixw}


%%%%%%%%%%%%%%%%%%%%%%%%%%%%%%%%%%%%%%%%%%%%%%%%%%%%%%%%%%%%%%%%%%%%%%%%%%%%%%
% Set headers

\usepackage{fancyhdr} 
\pagestyle{fancy}
\fancyhf{}
\fancyhead[C]{\begin{tabularx}{\textwidth}{@{}Xll}
\noalign{\vspace{1pt}}
\textbf{\Large Tara Sullivan} 
    & Email: &\ \myemail{} \\
\noalign{\vspace{1pt}}
Department of Economics 
    & Phone: &\ \myphone{} \\
\noalign{\vspace{1pt}}
University of California, San Diego 
    & Links: &\ \href{http://tara-sullivan.github.io}{Website}, 
    \href{https://github.com/tara-sullivan}{Github},
    \href{https://www.linkedin.com/in/tara-sullivan-econ/}{LinkedIn}
\end{tabularx}}

%%%%%%%%%%%%%%%%%%%%%%%%%%%%%%%%%%%%%%%%%%%%%%%%%%%%%%%%%%%%%%%%%%%%%%%%%%
% Format font of sections and lines for sections

\usepackage{titlesec} % Format section headers
% Add a horizontal line before each item
\titleformat{\section}{\titlerule\normalfont\large\bfseries}{\section}{1em}{}[]
% Alternative: add horizontal line after each item
% \titleformat{\section}{\normalfont\large\bfseries}{\section}{1em}{}[{\titlerule[0.4pt]}]

% Change title spacing
\titlespacing{\section}{0pt}{\parskip}{0pt}


%%%%%%%%%%%%%%%%%%%%%%%%%%%%%%%%%%%%%%%%%%%%%%%%%%%%%%%%%%%%%%%%%%%%%%%%%%%%%%
\begin{document} 

%%%%%%%%%%%%%%%%%%%%%%%%%%%%%%%%%%%%%%%%%%%%%%%%%%%%%%%%%%%%%%%%%%%%%%%%%%%%%%
\section*{Education}

\begin{tabularx}{\textwidth}{LR}
\textbf{University of California, San Diego}, La Jolla CA & Sept. 2015 - present
\\
Ph.D. Candidate in Economics & \emph{[Expected: June 2022]}
\end{tabularx} 
\emph{Primary fields:} Macroeconomics, Econometrics
\begin{blist}
\item Completed extensive coursework in causal inference, nonparametric and semiparametric models, and time series analysis 
\end{blist}
\emph{Dissertation:} ``Group-based beliefs and human capital specialization''
\begin{blist}
\item 
Explains persistent gender gaps in college major choice using an optimal stopping problem with Bayesian learning
% \item Intuitively, I theoretically formalize the following dynamic: if men and women choose their major based on their expected probability of success in a field, and those beliefs are formed based on existing outcomes for men and women, do otherwise identical men and women make different major choices? And how does this perpetuate gender gaps in college major choice?
\item Data analysis and theoretical modeling programmed in Python; see Github repository (\href{https://github.com/tara-sullivan/hcs}{link})
\end{blist}


\mysep{}
\begin{tabularx}{\textwidth}{@{}Xr@{}}
\textbf{Boston College}, Chestnut Hill, MA & Sept. 2008 - May 2012 \\
BA in Economics (Honors), BA in International Relations
\\
\emph{Phi beta kappa, magna cum laude, Giffuni prize for best senior thesis in the Economics department}
\end{tabularx}

%%%%%%%%%%%%%%%%%%%%%%%%%%%%%%%%%%%%%%%%%%%%%%%%%%%%%%%%%%%%%%%%%%%%%%%%%%%%%%%%
\mysep{}
\section*{Employment history}

\begin{tabularx}{\textwidth}{LR}
\textbf{Wayfair}, Boston, MA
&
June 2021 - August 2021
\\
\emph{Economist Intern}
\end{tabularx}
\begin{blist}
\item Evaluated the performance of deepAR, a global deep learning model, in producing demand forecasts
\item Theoretically analyzed the effectiveness of global models more broadly for different forecasting workflows
\item Python packages used include PyTorch, GluonTS, TensorBoard and PySpark. Worked in GCP
\end{blist}

\mysep{}
\begin{tabularx}{\textwidth}{LR}
\textbf{University of California, San Diego}, La Jolla CA 
&
Oct. 2015 - present
\\
\emph{Teaching Assistant}
\end{tabularx}

\begin{blist}
% \item 

\item Extensive experience teaching the critical analysis of statistical models, from a theoretical and applied perspective, across different education levels (undergraduate, masters, PhD), and departments (Economics, Public Policy, Data Science)
\begin{blist}
% \item Theoretical topics taught (quarters taught in parenthesis): Probability and statistics (x3); linear regression (x6); panel data analysis (x3); linear programming (x2); data visualization (x4); batch scripts and server usage (x2)
\item Theoretical topics taught: probability and statistics (PhD level), linear regression, panel data analysis, introductory data analysis, linear programming, data visualization, causal inference
\item Applications taught: 
    economics of discrimination, 
    fairness in algorithmic decision making, 
    % environmental economics, 
    international economics
\end{blist}

% \item Adept at explaining statistics to a wide variety of audiences, including the basics of probability, statistics, and linear regression to undergraduates; policy-relevant econometrics to Master's students; and measure theory, statistical inference, and computation to first year Economics PhD students

\item Managed a teaching staff of 10-12 for a Principals of Macroeconomics, a course with 500 students. Organized transition to online exams in March 2020, and to full online instruction in 2021

\item Recommended as a TA by 95\% of students; won 2019 TA Excellence award

% \item Principals of Macroeconomics, Winter 2019, 2020, \& 2021. Managed a teaching staff of 10-12 for a course of 500 students. Organized transition to online exams in March 2020, and to full online instruction in 2021. Instructor: Valerie Ramey. 

% \item Ph.D. Econometrics A, Fall 2017 \& 2018. Taught first-year economics graduate students probability theory and statistical inference. Subsequently won 2019 TA Excellence award. Instructors: Brendan Beare \& Graham Elliot.

% \item Ph.D. Computation, Winter 2017 \& 2018. Taught first-year economics graduate students data visualization in Stata, dynamic programming in Matlab, as well as basic batch scripts and server usage. Instructors: Michelle White \& Garey Ramey.

% \item M.A. International Economics, Spring 2018, 2019, \& 2020. Taught public policy master's students data cleaning in Stata, regression analysis, and structural modeling. Instructors: Natalia Ramondo \& Renee Bowen.

\end{blist}

\mysep{}
\begin{tabularx}{\textwidth}{LR}
\textbf{Environmental Defense Fund}, New York, NY 
&
July 2019 - Sept. 2019
\\
\emph{Pre-doctoral Intern}, Office of the Chief Economist
\end{tabularx}
\begin{blist}
\item Wrote forthcoming policy report summarizing how spatial general equilibrium models and time series analysis techniques can be used to understand agricultural climate change adaptation.
\end{blist}

\mysep{}
\begin{tabularx}{\textwidth}{LR}
\textbf{University of California, San Diego}, La Jolla, CA 
& June 2018 - Nov. 2018
\\
\emph{Research Assistant}
\end{tabularx}
\begin{blist}
\item Wrote Structural VAR ado files in Stata to estimate impulse response functions that maximize forecast error variance. Required extensive use of Mata and knowledge of MATLAB. Supervisor: Valerie Ramey
\item Analysis of the effectiveness of fixed vs random effects for macroeconomic model identification. Supervisor: David Lagakos
\end{blist}

\mysep{}
\begin{tabularx}{\textwidth}{LR}
\textbf{Federal Reserve Bank of New York}, New York, NY
& June 2012 - July 2015 
\\
\emph{Senior Research Analyst}, Research Group, Financial Intermediation
\end{tabularx}
\begin{blist}
\item Overhauled system through which the Research, Markets, and Banking Supervision groups access bank holding company regulatory data using SQL, SAS, and Stata
\item Produced quarterly policy reports on banking industry using Stata and VBA
\end{blist}

%%%%%%%%%%%%%%%%%%%%%%%%%%%%%%%%%%%%%%%%%%%%%%%%%%%%%%%%%%%%%%%%%%%%%%%%%%%%%%
\mysep{}
\section*{Technical skills}

\textbf{Computing skills}
\begin{blist}
\item Primary languages:
Python
(\href{https://github.com/tara-sullivan/hcs/blob/master/hcs/model/afmodel.py}{quantitative example}; 
\href{https://github.com/tara-sullivan/hcs/blob/master/hcs/img/code/plot_line_labels.py}{matplotlib example});
Stata (\href{https://github.com/tara-sullivan/hcs/blob/master/hcs/data/ipeds/c/clean_data/ipeds_c_clean.do}{example}); 
\LaTeX; Git; SQL; MATLAB
%Mata, VBA, SAS

\item Packages:
pandas, NumPy, Matplotlib, statsmodels, scikit-learn, GluonTS, PyTorch, TensorFlow
\item Systems: MacOS, Windows, Linux, GCP
\end{blist}

\mysep{}
\textbf{Analysis}
\begin{blist}
\item Fluent:
Statistical inference (hypothesis testing, interval estimation);
Point estimation methods (MLE, GMM, Bayes estimators);
Linear regression;
Time-series analysis;
Forecasting;
% Time-series analysis (reduced-form and structural VARs, cointegration, forecast error variance decomposition);
Optimal stopping problems;
Deep learning;
Panel data analysis
% Panel data analysis (unobserved effects models, discrete choice models, and censored and truncated regression models);
% Instrumental variables;
% Nonparametric and semi-parametric models (kernel density estimation, bandwidth selection and cross validation, local polynomial regression);

\item Familiar: 
% General method of sieves;
% Bayesian networks;
Observational causal inference;
Regression discontinuity;
Potential outcomes models;
% Markov models;
Propensity score matching; 
% Bootstrapping;
% Synthetic control methods; 
Principal component analysis;
Quantile regression;
Numerical linear algebra;
Nonlinear systems and numerical optimization 
\end{blist}

\end{document}
